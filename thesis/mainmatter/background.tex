\chapter{Background}
\label{ch:background}
In the following chapter, we introduce all significant entities and tools adopted by the MMK Driving Simulator. The simulator itself is implemented in Unity3D, while all of the 3D models and textures are generated by ESRI CityEngine. Additionally, the map information is exported from OpenStreetMap and the traffic simulation is done using SUMO Engine. We discuss essential characteristics of these tools which are later used to achieve the goals described in Chapter \ref{ch:introduction}.

\section{OpenStreetMap}
\label{ch:osm}
OpenStreetMap (OSM) is a open source project that creates and distributes free geographic data for the world. The project was created by the OpenStreetMap Foundation and it is completely supported by volunteers. OSM represents physical features on the ground (\emph{e.g.}, roads or buildings) using tags attached to its basic data structures (its \texttt{nodes}, \texttt{ways}, and \texttt{relations}). Each tag describes a geographic attribute of the feature being shown by that specific node, way or relation. Moreover, each basic data structure of OSM has a 64 bit integer identification number (\emph{OSM ID}), which allows every object to be uniquely identifiable.

A \emph{node} represents a specific point on the earth's surface defined by its latitude and longitude. They can be used to define standalone point features \emph{e.g.} park bench or a water well or they could define the shape of a way. A \emph{way} is an ordered list of between 2 and 2,000 nodes that define a polyline. Ways are used to represent linear features such as rivers and roads, as well as boundaries of areas such as buildings. Finally, a \emph{relation} is a multi-purpose data structure that documents a relationship between two or more data elements, \emph{e.g.}  route relation.

There exist a possibility to export a certain area of the map in an \texttt{xml} format which can be easily parsed. An example export can be seen in Figure[*] OSM uses the WGS-84 coordinate system, \emph{i.e.} each map object obtains a latitude and longitude value. 

\section{ESRI City Engine}
\label{ch:ce}
\emph{ESRI CityEngine} (CE) is a three-dimensional modelling software application developed by \emph{Esri R\&D Center Zurich} and is specialised in the generation of 3D urban environments. There are multiple possibilities to create a city model with CE. One possibility is to import a third-party (2D) map data, such as OSM described in the previous section or 3D models in \texttt{.shp} or \texttt{.fbx} format. Afterwards, the application will try to generate 3D models using the 2D data and additional information (\emph{e.g.} defined tags for every OSM basic data structure) and specified by the user rules. Finally, the generated shapes can be exported to commonly used formats such as \texttt{.obj} or \texttt{.fbx}. 

The coordinate system used to represent the location of all objects in the scene is a custom variation of the Universal Transverse Mercator coordinate system (UTM). A position on the scene is given by easting and northing values as in UTM. However, instead of a zone number, easting and northing signs are determined according to their side according to the \emph{Prime meridian} and the \emph{Equator}. When the user imports some data to CE, each models coordinates are translated to the CE coordinate system. This transformation is reversible by using for example [*] python module. When exporting 3D models from CE to one of the compatible formats, one could center the whole scene automatically (as can be seen in Figure [*]), so the origin of all models is located at \texttt{(0, 0, 0)}.

The generated street network model is composed of \texttt{segments} and \texttt{nodes}. Each node is defined by one 3D point in the scene, while each segment composes of exactly two nodes which define its form. Both, nodes and segments, have unique ID (\emph{OID}) and can hold shapes as children objects, which are referenced with the same ID as their parent segment or node and an index (\emph{OID:INDEX}). [*] has already implemented a python script in his research project at the Institute of Human-Machine Interaction, TUM which exports all nodes, segments and their corresponding shapes in JSON format. 

CE includes a \texttt{Jython} interface which can be utilised to access and change scene's objects properties. CE 2016 uses \texttt{Jython 2.7b21},  which already has the most common Python modules, however there are some necessary software which has to be manually included to the \texttt{Jython} environment. In order to use the export scripts which will be described in the next chapter, one have to download and copy \texttt{simplejson} folder to \texttt{C:$\backslash$Users$\backslash$<USERNAME>$\backslash$.CityEngine$\backslash$2016.1R.win32.win32.x86\_64$\backslash$jythonCache$\backslash$\\ThirdParty}.

\section{Simulation of Urban MObility (SUMO)}
\label{ch:sumo}

\section{Unity}
\label{ch:unity}
